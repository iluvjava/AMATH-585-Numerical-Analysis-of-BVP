\documentclass[]{article}
\usepackage{amsmath}\usepackage{amsfonts}
\usepackage[margin=1in,footskip=0.25in]{geometry}
\usepackage{mathtools}
\usepackage{hyperref}
\hypersetup{
    colorlinks=true,
    linkcolor=blue,
    filecolor=magenta,
    urlcolor=cyan,
}
\usepackage[final]{graphicx}
\usepackage{listings}
\usepackage{courier}
\lstset{basicstyle=\footnotesize\ttfamily,breaklines=true}
\newcommand{\indep}{\perp \!\!\! \perp}
% \usepackage{wrapfig}
\graphicspath{{.}}
% \usepackage{fancyvrb}

%%
%% Julia definition (c) 2014 Jubobs
%%
\usepackage[T1]{fontenc}
\usepackage{beramono}
\usepackage[usenames,dvipsnames]{xcolor}
\lstdefinelanguage{Julia}%
  {morekeywords={abstract,break,case,catch,const,continue,do,else,elseif,%
      end,export,false,for,function,immutable,import,importall,if,in,%
      macro,module,otherwise,quote,return,switch,true,try,type,typealias,%
      using,while},%
   sensitive=true,%
   alsoother={$},%
   morecomment=[l]\#,%
   morecomment=[n]{\#=}{=\#},%
   morestring=[s]{"}{"},%
   morestring=[m]{'}{'},%
}[keywords,comments,strings]%

\lstset{%
    language         = Julia,
    basicstyle       = \ttfamily,
    keywordstyle     = \bfseries\color{blue},
    stringstyle      = \color{magenta},
    commentstyle     = \color{ForestGreen},
    showstringspaces = false,
}
\begin{document}
\begin{center}
    Name: Hongda Li
    \\
    AMATH 585 WINTER 2022 HW 6
\end{center}
\section*{Problem 1}
    Any function with chebyshev coefficients $a_0, a_1, \cdots, a_n$, evaluated at the chbyshev node is given as: 
    \begin{align*}\tag{1}\label{eqn:1}
        p(\cos(k\pi/n)) &= 
        \sum_{j = 0}^{n}a_j\cos(jk\pi/n)
    \end{align*}
    Our objective here is make use hf the FFT algorithm for DFT for the objective of: Interpolation the function at chebyshev node getting the values of $a_0, a_1, \cdots, a_n$, and evaluating the function value at the chebyshev nodes using the FFT algorithm. 
    \par
    My claim here is that, if we tiled the vector in the following format: $[a_0, a_1, \cdots, a_{n - 1}, a_n, a_{n - 1}, \cdots, a_1]$, so that it's symmetric exclusing the first element, and then we put this into the DFT algorithm using FFT, then we obtain the following relationsship: 
    


\end{document}