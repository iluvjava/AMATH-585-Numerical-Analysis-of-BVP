\documentclass[]{article}
\usepackage{amsmath}\usepackage{amsfonts}
\usepackage[margin=1in,footskip=0.25in]{geometry}
\usepackage{mathtools}
\usepackage{hyperref}
\hypersetup{
    colorlinks=true,
    linkcolor=blue,
    filecolor=magenta,
    urlcolor=cyan,
}
\usepackage[final]{graphicx}
\usepackage{listings}
\usepackage{courier}
\lstset{basicstyle=\footnotesize\ttfamily,breaklines=true}
\newcommand{\indep}{\perp \!\!\! \perp}
% \usepackage{wrapfig}
\graphicspath{{.}}
% \usepackage{fancyvrb}

%%
%% Julia definition (c) 2014 Jubobs
%%
\usepackage[T1]{fontenc}
\usepackage{beramono}
\usepackage[usenames,dvipsnames]{xcolor}
\lstdefinelanguage{Julia}%
  {morekeywords={abstract,break,case,catch,const,continue,do,else,elseif,%
      end,export,false,for,function,immutable,import,importall,if,in,%
      macro,module,otherwise,quote,return,switch,true,try,type,typealias,%
      using,while},%
   sensitive=true,%
   alsoother={$},%
   morecomment=[l]\#,%
   morecomment=[n]{\#=}{=\#},%
   morestring=[s]{"}{"},%
   morestring=[m]{'}{'},%
}[keywords,comments,strings]%

\lstset{%
    language         = Julia,
    basicstyle       = \ttfamily,
    keywordstyle     = \bfseries\color{blue},
    stringstyle      = \color{magenta},
    commentstyle     = \color{ForestGreen},
    showstringspaces = false,
}
\begin{document}
\begin{center}
    Name: Hongda Li
    \\
    AMATH 585 HW4 2022 WINTER
\end{center}
\section*{Problem 1}
    Consider the boundary value problem: 
    $$
        \begin{cases}
            -\frac{d}{dx}\left(
                (1 + x^2)\frac{du}{dx}
            \right) = f(x) & \forall x \in [0, 1] 
            \\
            u(0) = u(1) = 0 & 
        \end{cases}
    $$
    \subsection*{(a)}
        \hspace{1.1em}
        \textbf{Objective}: Derive the Matrix equation that we will need to solve the problem, using Galerkin Finite Element Method, With continuous piecewise linear basis functions. 
        \par
        Introduce $\mathcal{L}[\cdot]:= \partial_x[r(x)\partial[\cdot]]$ as a differential operator, to derive, consider the weak form: 
        \begin{align*}\tag{1.a.1}\label{eqn:1.a.1}
            \langle \mathcal{L}[u], \varphi\rangle 
            &\equiv 
            \int_{0}^1
            (\partial_x[r(x)\partial_x[u(x)]])\varphi(x)dx 
            = 
            \int_{0}^{1} f(x)\varphi(x)dx \equiv \langle f,\varphi\rangle
            \\
            &= 
            \int_{0}^{1} 
                \varphi(x)d(r(x)\partial_x[u(x)])
            dx
            \\
            &= 
            \left.\varphi(x)(r(x)\partial_x[u(x)])\right|_0^1
            -
            \int_{0}^{1} 
                r(x)\partial_x[u(x)]\varphi'(x)
            dx
            \\
            &= 
            \varphi(1)r(1)\partial_x[u](1) - \varphi(0)r(0)\partial_x[u](0) - 
            \int_{0}^{1} 
                r(x)\partial_x[u(x)]\varphi'(x)
            dx
        \end{align*}
        Where, $\varphi(x)$ is some basis function in $\mathcal{S}$, in our cause, $\mathcal{S}$ is finite and has $n - 1$ elements (Because it's constructed on an discretized grid points), and let $\varphi_i(x)$ be the index basis function, then we have: 
        \begin{align*}\tag{1.a.2}\label{eqn:1.a.2}
            \varphi_i(x) &= 
            \begin{cases}
                \frac{x - x_{i - 1}}{x_i - x_{i - 1}} & x\in [x_{i - 1}, x_i]
                \\
                \frac{x_{i + 1} - x}{x_{i + 1} - x_{i}} & x\in [x_{i}, x_{i + 1}]
                \\
                0 & \text{else}
            \end{cases}
        \end{align*}
        The objective of the finite elements method is to represents the solution, $u(x)$ as a linear combinations of the basis function $\varphi_i(x)$. Suppose that $\hat{u} = \sum_{j = 1}^{n - 1}c_j \varphi_j(x)$, then reconsidering \hyperref[eqn:1.a.1]{(1.a.1)} the weak form and convert it to a matrix vector equation: 
        \begin{align*}\tag{1.a.3}\label{eqn:1.a.3}
        \end{align*}



\end{document}