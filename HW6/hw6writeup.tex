\documentclass[]{article}
\usepackage{amsmath}\usepackage{amsfonts}
\usepackage[margin=1in,footskip=0.25in]{geometry}
\usepackage{mathtools}
\usepackage{hyperref}
\hypersetup{
    colorlinks=true,
    linkcolor=blue,
    filecolor=magenta,
    urlcolor=cyan,
}
\usepackage[final]{graphicx}
\usepackage{listings}
\usepackage{courier}
\lstset{basicstyle=\footnotesize\ttfamily,breaklines=true}
\newcommand{\indep}{\perp \!\!\! \perp}
% \usepackage{wrapfig}
\graphicspath{{.}}
% \usepackage{fancyvrb}

%%
%% Julia definition (c) 2014 Jubobs
%%
\usepackage[T1]{fontenc}
\usepackage{beramono}
\usepackage[usenames,dvipsnames]{xcolor}
\lstdefinelanguage{Julia}%
  {morekeywords={abstract,break,case,catch,const,continue,do,else,elseif,%
      end,export,false,for,function,immutable,import,importall,if,in,%
      macro,module,otherwise,quote,return,switch,true,try,type,typealias,%
      using,while},%
   sensitive=true,%
   alsoother={$},%
   morecomment=[l]\#,%
   morecomment=[n]{\#=}{=\#},%
   morestring=[s]{"}{"},%
   morestring=[m]{'}{'},%
}[keywords,comments,strings]%

% \lstset{%
%     language         = Julia,
%     basicstyle       = \ttfamily,
%     keywordstyle     = \bfseries\color{blue},
%     stringstyle      = \color{magenta},
%     commentstyle     = \color{ForestGreen},
%     showstringspaces = false,
% }
\begin{document}
\begin{center}
    Name: Hongda Li
    \\
    AMATH 585 WINTER 2022 HW 6
\end{center}
\section*{Problem 1}
    Any function with chebyshev coefficients $a_0, a_1, \cdots, a_n$, evaluated at the chbyshev node is given as: 
    \begin{align*}\tag{1}\label{eqn:1}
        p(\cos(k\pi/n)) &= 
        \sum_{j = 0}^{n}a_j\cos(jk\pi/n)
    \end{align*}
    Our objective here is make use hf the FFT algorithm for DFT for the objective of: Interpolation the function at chebyshev node getting the values of $a_0, a_1, \cdots, a_n$, and evaluating the function value at the chebyshev nodes using the FFT algorithm. It's implied that $k, n$ are integers in this context. 
    \par
    My claim here is that, if we tiled the vector in the following format:
    $$\vec{f} = [a_0, a_1, \cdots, a_{n - 1}, a_n, a_{n - 1}, \cdots, a_1]$$
    It's symmetric exclusing the first element, and then we put this into the DFT algorithm using FFT, then we obtain the following relationsship: 
    \begin{align*}\tag{1.1}\label{eqn:1.1}
        \frac{1}{2}(F_k + a_0 + (-1)^k)
        &= 
        \sum_{j = 0}^{n}\cos\left(
            \frac{\pi j k}{n}
        \right)a_j \quad \forall 0 \le k \le n
    \end{align*}
    Here, we assume that the vector $\vec{F} = [F_0, F_1, \cdots F_{2n - 1}]$ are the output of the DFT after we feed $\vec{f}$ into the algorithm. 
    \par
    Before we prove \hyperref[eqn:1.1]{(1.1)} we wish to establish some basics about the vector $\vec{f}$. Observe that the vector is symmetric if we exclude the first argument, which means that $f_j = f_{2n - j}\; \forall\; 1 \le j \le 2n - 1$. Next, the vector $\vec{f}$ has a total length of $2n$. And when we index the vector $\vec{f}, \vec{F}$, we let the \textbf{index starts with zero}. 
    \par
    First, consider the following algebra: 
    \begin{align*}\tag{1.2}\label{eqn:1.2}
        \exp\left(
            -i\frac{\pi(2n - j)k}{n}
        \right) &= 
        \exp\left(
            -i \frac{2\pi n - j\pi k}{n}
        \right)
        \\
        &= \exp
        \left(
            -i\frac{2n\pi n}{n} + \frac{ij\pi k}{n}
        \right)
        \\
        &= \exp\left(
            i\frac{jk\pi}{n}
        \right)
        \\
        \sum_{j = 0}^{2n - 1}
        \exp\left(
            -i \frac{i\pi j k}{n}
        \right) &= 
        \sum_{j = 1}^{2n}
        \exp\left(
            -i \frac{2\pi(2n - j)k}{n}
        \right)
    \end{align*}
    The second equality is just a trick where I swapp the index so it starts summing in the reverse order. 
    \par
    Now consider the DFT on vector $\vec{f}$, which by definition would be given as: 
    \begin{align*}\tag{1.3}\label{eqn:1.3}
        F_k &= \sum_{j = 0}^{2n - 1}
            \exp\left(
                - i \frac{2\pi j k}{2n}
            \right)f_j
        = \sum_{j = 0}^{2n - 1}
        \exp\left(
            - i \frac{\pi j k}{n}
        \right)f_j
        \\
        &= 
        \frac{1}{2}
        \left(
            \sum_{j = 0}^{2n - 1}
            \exp
            \left(
                -i\frac{\pi j k}{n}
            \right)f_j
            + 
            \sum_{j = 1}^{2n}
            \exp\left(
                -i \frac{2\pi (2n - j)k}{n}
            \right)\underbrace{f_{2n -j}}_{= f_j}
        \right)
        \\
        &= 
        \frac{1}{2}
        \left(
            \sum_{j = 0}^{2n - 1}
            \exp
            \left(
                -i\frac{\pi j k}{n}
            \right)f_j
            + 
            \sum_{j = 1}^{2n}
            \exp\left(
                i\frac{jk\pi}{n}
            \right)f_j
        \right) \impliedby \quad \text{by: \hyperref[eqn:1.2]{(1.2)}}
        \\
        &= \frac{1}{2}
        \left(
            2f_0 + 
            \sum_{j = 1}^{2n - 1}
            \exp
            \left(
                -i\frac{\pi j k}{n}
            \right)f_j
            + 
            \sum_{j = 1}^{2n - 1}
            \exp\left(
                i\frac{jk\pi}{n}
            \right)f_j
        \right)
        \\
        &= f_0 + \sum_{j = 1}^{2n - 1}
        \cos\left(
            \frac{\pi j k}{n}
        \right)f_j
    \end{align*}
    Next, please observe the fact that the term for $j = 1$ equals to $j = 2n - 1$, due to the symmetry of $\cos$ and the symmetry of vector $f_j\; \forall 1 \le j \le 2n - 1$. And hence we obtained: 
    \begin{align*}\tag{1.4}\label{eqn:1.4}
        F_k &= a_0 + \left(
            2 \sum_{j = 1}^{n - 1}
            \cos\left(
                \frac{\pi j k}{n}
            \right)a_j
        \right) + (-1)^k a_n
    \end{align*}    
    Here, take note of the extra term, when $j = n$, $f_j = n$, which is right in the middle of the symmetric part of $\vec{f}$, and it only repeats once, so I take it out from the sum and it produces the term $(- 1)^ka_n$. All other terms repeats 2 times and $f_0 = a_0$. Rearranging the above equation we have: 
    \begin{align*}\tag{1.5}\label{eqn:1.5}
        \frac{1}{2}
        \left(
            F_k - a_0 - (-1)^ka_n
        \right) &= 
        \sum_{j = 1}^{n - 1}
        \cos \left(
            \frac{\pi j k}{n}
        \right)a_j
        \\
        \frac{1}{2}
        \left(
            F_k + a_0 + (-1)^ka_n
        \right) &= 
        \sum_{j = 0}^{n}
        \cos \left(
            \frac{\pi j k}{n}
        \right)a_j
        \\
        \frac{1}{2}
        \left(
            F_k + a_0 + (-1)^ka_n
        \right) &= p\left(
            \cos\left(
                \frac{k\pi}{n}
            \right)
        \right)
    \end{align*}
    From the frist line to the second line, I added $a_0, (-1)^ka_n$ to both side of the equation. At this point, we have proven that \hyperref[eqn:1.1]{(1.1)} is true, and we can make use of the algorithm fast evaluate the chebyshev series at the chebyshev nodes. Simply make the vector $\vec{f}$ as said above, and then evalute it to get $F_k$, and then use that above expression, for $k = 0, \cdots, n$. There will be $2n$ output vectors, but we can ignore the part where it gets symmetric. 
    \par
    Next, to reverse the process for looking for the chebyshee coefficients, we simply consider: "What is $F_k$"? And then make use of the IDFT algorithm which uses IFFT. 
    \begin{align*}\tag{1.6}\label{eqn:1.6}
        p\left(
            \cos\left(
                \frac{k\pi}{n}
            \right)
        \right) &= 
        \frac{1}{2}
        \left(
            F_k + a_0 + (-1)^ka_n
        \right)
        \\
        2 p\left(
            \cos \left(
                \frac{k\pi }{n}
            \right)
        \right)  
        &= F_k + a_0 + (-1)^ka_n
        \\
        2 p\left(
            \cos \left(
                \frac{k\pi }{n}
            \right)
        \right) - a_0 - (-1)^ka_n
        &= F_k
    \end{align*}
    Do this for $k = 0, \cdots, 2n - 1$ and then invoke the IFDT using FFT, and then we get back the veoctr $\vec{f}$, and the first $n + 1$ elements are the chbyshev coefficients. 





\end{document}